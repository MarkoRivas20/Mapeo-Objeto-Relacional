\section{Objetivos} 
\textbf{}\\
\begin{flushright}


\begin{itemize}
El desarrollo de software busca mejorar la productividad de las empresas por medio de la automatización y el uso de herramientas, las empresas que se dedican al desarrollo de software tienen como objetivo ayudar a otras empresas en la automatización y desarrollo de herramientas para este fin, pero muchas veces las mismas olvidan su propia productividad. Las Herramientas ORM (Object  Relational  Mapping) se han ideado con este fin, al evitar repetir muchas líneas de programación en la capa de abstracción, pero es necesario considerar que cada una tiene su estándar o su propio lenguaje por lo cual se requeriría tiempo adicional para ocuparlas.  \textbf{}\\

El desarrollo de este ORM tiene como objetivo proveer una herramienta acorde a las necesidades y el estándar de programación de la empresa Interfaces, logrando  que durante el desarrollo de software, este se centre en otros aspectos más relevantes, como el diseño de la base de datos, interfaz de usuario y la capa de negocios, olvidándonos casi por completo de la capa de abstracción. Con el uso de la herramienta ORM se logró reducir el tiempo de desarrollo de la aplicación de prueba. \textbf{}\\

Realizar el modelo y modo de implementación del ORM en un java-object. \textbf{}\\
Implementar la recuperación de datos automáticamente de los java-objects. \textbf{}\\
Implementar la conectividad y las consultas autogeneradas a la Base de Datos a partir de los modelos heredados del modelo base del ORM. \textbf{}\\
Implementar la conversión automática de los resultados a los modelos java-object. \textbf{}\\
Realizar pruebas de validación al ORM. \textbf{}\\



\end{itemize} 


\end{flushright}