\section{Conclusiones} 
\begin{flushleft}
\begin{itemize}
\textbf{}\\
El uso de un ORM es una alternativa sumamente efectiva a la hora de trasladar el modelo conceptual
(orientado a objetos) al esquema relacional nativo de las bases de datos SQL. Evita la inclusión de
sentencias SQL embebidas en el código de la aplicación, lo que a su vez facilita la migración hacia otro
sistema gestor de bases de datos. Incorpora una capa de abstracción entre el modelo relacional físico y
la capa de negocios de la aplicación. Al ser realizado, en esta capa, de manera automática la conversión
de instrucciones orientadas a objetos, a sentencias SQL, minimiza la ocurrencia de errores humanos.

\textbf{}\\
De cualquier modo utilizar un ORM no debe ser considerado una panacea, sino que debe usarse a
discreción; teniendo en cuenta las particularidades de cada problema a modelar. En determinados casos
no es recomendable el uso de un ORM, sobre todo cuando se imponen tiempos de respuesta mínimos o
se requiere una menor sobrecarga. En estos casos lo más conveniente es el uso de un microORM;
evitando siempre que sea posible las inyecciones de SQL Inline.

\textbf{}\\
Lo anteriormente expuesto libera a los desarrolladores de aplicaciones de la responsabilidad de conocer
las múltiples variantes de SQL que existen en función del gestor de bases de datos que se utilice. No
obstante, en escenarios en que se necesite hacer un uso más eficiente del sistema de almacenamiento
de información, personalizado de acuerdo a las necesidades de la aplicación, y se escoja como gestor de
bases de datos una variante NoSQL no es necesaria la utilización de un ORM. 

\textbf{}\\
En resumen, en dependencia del gestor de bases de datos a emplear, se recomienda siempre que sea
posible y sea factible la utilización de un ORM


\end{itemize} 


\end{flushleft}