\section{Análisis (apreciaciones)} 
\textbf{}\\
\begin{flushleft}


\begin{itemize}
Comencemos analizando JDBC(API de Java)como solución. El mayor beneficio es el rendimiento dado
que programado adecuadamente, con store procedures y al no tener capas intermedias
que pasar, el rendimiento aumenta a sacrificio de muchas características que son de suma
importancia en una arquitectura empresarial. \textbf{}\\
JDBC es una API de bajo nivel y sus prestaciones son para funciones de bajo nivel.
Cuando el modelo de datos es simple la solución más fácil es JDBC \textbf{}\\
Los frameworks ORM ya están consolidados, probados en el mercado y existen muchas
herramientas para desarrollar con estos frameworks. La mayoría de ellos permiten la
definición de las entidades generalmente a través de ficheros XML o anotaciones. A partir
de esa definición los ORM son capaces de extraer la definición de esquema de la base de
datos necesaria para representar ese modelo de objetos. Los ORM permiten el mapeo de
estos esquemas de base de datos a objetos de modo que a partir de las tablas de base de
datos se pueden generar las clases Java, necesarias para modelar dicho esquema con
todas las relaciones de jerarquía que estén presentes en el mismo.\textbf{}\\
Obviamente esto supone una ventaja grandísima ya que la cantidad de código necesaria
para realizar todas esas funciones es realmente considerable\textbf{}\\
\end{itemize} 


\end{flushleft}
